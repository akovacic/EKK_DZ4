\documentclass[12pt, a4paper]{article}
\usepackage[utf8]{inputenc}
\usepackage[croatian]{babel}
\usepackage[margin=1in]{geometry}
\usepackage{amsmath}
\usepackage{amssymb}
\usepackage{amsthm} 
\usepackage{mathtools}
\usepackage{hyperref}
\usepackage{graphicx}
\usepackage{multirow}
\usepackage{fancyhdr} % prored, numeriranje stranice...
\usepackage{amsfonts}
\usepackage{setspace}% za postavljanje praznina
\usepackage{color}
\usepackage{wasysym}
\usepackage{bbm}
\usepackage{lscape}
\usepackage{filecontents}
\usepackage{multicol}
\usepackage{url}
\DeclareMathOperator {\sg}{sg}
\fancyhf{}
\renewcommand{\headrulewidth}{0pt}
\rfoot{\thepage}
\usepackage{paralist}
\usepackage{color}
\usepackage{tikz}
\usetikzlibrary{arrows,positioning,automata}
\PassOptionsToPackage{usenames,dvipsnames,svgnames}{xcolor}
 \newcommand{\HRule}{\rule{\linewidth}{0.5mm}}
\setlength\parindent{24pt}
\DeclarePairedDelimiter\floor{\lfloor}{\rfloor}
\DeclarePairedDelimiter\ceil{\lceil}{\rceil}
\usepackage{listings}
\usepackage{xcolor}
\lstset { %
    language=C++,
    backgroundcolor=\color{black!5}, % set backgroundcolor
    basicstyle=\footnotesize,% basic font setting
}

\begin{document}
\renewcommand{\bibname}{Literatura}%ne želim da mi piše bibliografija već literatura
\begin{titlepage}
\begin{center}
\newtheorem{thm}{Teorem}[section] %u svakom poglavlju teoremi se numeriraju zasebno
\theoremstyle{definition}
\newtheorem{defn}[thm]{Definicija} %enumeracija definicija jednaka enum teorem
\newtheorem{exmp}[thm]{Primjer} % analogno s primjerima
\newtheorem{nap}[thm]{Napomena} % analogno s naomenom

\LARGE Sveučilište u Zagrebu \\[1cm] Prirodoslovno - matematički fakultet \\[1cm] Matematički odsjek\\[3cm]

\Large Eliptičke krivulje u kriptografiji \\[0.5cm]

% Title
\HRule \\[0.4cm]
{ \huge \bfseries 4. domaća zadaća}\\[0.4cm]

\HRule \\[9cm]

% Autor i mentor -> postavljanje ovoga na dno stranice...
\begin{minipage}{0.4\textwidth}
\begin{flushleft} \large
\emph{Student:\\}
Antonio Kovačić
\end{flushleft}
\end{minipage}
\begin{minipage}{0.4\textwidth}
\begin{flushright} \large
\emph{Nastavnik:\\}
Doc. dr. sc. Filip Najman
\end{flushright}
\end{minipage}


\vfill

% Bottom of the page
{\large Zagreb, \today}

\end{center}
\end{titlepage}
\pagestyle{plain} % No headers, just page numbers
\pagenumbering{roman} % Roman numerals
\setcounter{page}{1}
\begingroup
\let\cleardoublepage\relax
\tableofcontents

\newpage


\begin{spacing}{1.5}
\section{Problem}
\begin{enumerate}
\item Zadana je točka $P=(0,2)$ na eliptičkoj krivulji $y^2=x^3+2x+4$ nad poljem $\mathbb{F}_{211}$. Odredite NAF prikaz broja $106$. Izračunajte $106P$.
\item Pronađite jednu eliptičku krivulju $E$ nad $\mathbb{F}_{17}$ sa svojstvom da je red grupe $E(\mathbb{F}_{17})$ jednak 20.
\item Zadana je eliptička krivulja
\[E \, : \; y^2=x^3+x+4  \]
nad poljem $\mathbb{F}_{191}$. Odredite red grupe $E(\mathbb{F}_{191})$ Shanks-Mestreovom metodom, koristeći točku $P=(24,10)$.
\end{enumerate}
\newpage
\section{Rješenje}
\subsection{1. zadatak}
Za prvi zadatak sam koristio $Sage$ i $Python$. Rješenje je isprogramirano u $Sageu$. U \cite[s. ~58-59]{ekk} su navedeni algoritmi za traženje zadanih informacija. No kako bi se sve poklopilo s indeksima, na binarni zapis sam dodao još 2 nule (jer u algoritmu za NAF prikaz petlja ide od $0$ do $d$, pa fale još dvije znamenke ispred jer $n_d$ i $n_{d+1}$ nisu definirani bili. U konačnici, implementacija je dana sa:
\lstinputlisting[language=Python]{EKKDZ4_1.py}
U zadnje dvije linije vidimo ispis rješenja. Rješenja su:\\
NAF prikaz je $s=(0,1,0,1,0,-1,0,1)$, a $106P=(2,4)$
\newpage
\subsection{2. zadatak}
Ovaj zadatak sam riješio primjenom tehnike \textit{brute forcea}. U \textit{sageu} sam kreirao krivulje oblika $y^2=x^3+ax+b$ čija je diskriminanta različita od 0, $a,b \in \mathbb{F}_{17}$ su proizvoljni. Zatim sam samo provjeravao je li $|E(\mathbb{F}_{17})|=20$. Druga je opcija bila da iz formule za računanje ranga:
\[|E(\mathbb{F}_{17})|=17+1+\sum_{x \in \mathbb{F}_{17}} \left( \frac{x^3+ax+b}{17} \right)=20\]
nađem $a$ i $b$ takve da je $\sum_{x \in \mathbb{F}_{17}} \left( \frac{x^3+ax+b}{17} \right)=2$ što se na kraju svodi na isto.\footnote{Izraz  $\left( \frac{x^3+ax+b}{17} \right)$ je Legendreov simbol definiran u \cite[s. ~29]{utb}} Implementacija izrečenog algoritma slijedi:
\lstinputlisting[language=Python]{EKKDZ4_2.py}
Program je kao riješenja dao dole navedene krivulje:
\[y^2 = x^3 + x + 6\]
\[y^2 = x^3 + x + 11\]
\[y^2 = x^3 + 2x\]
\[y^2 = x^3 + 3x + 4\]
\[y^2 = x^3 + 3x + 13\]
\[y^2 = x^3 + 4x + 3\]
\[y^2 = x^3 + 4x + 14\]
\[y^2 = x^3 + 5x + 8\]
\[y^2 = x^3 + 5x + 9\]
\[y^2 = x^3 + 6x + 7\]
\[y^2 = x^3 + 6x + 10\]
\[y^2 = x^3 + 7x + 5\]
\[y^2 = x^3 + 7x + 12\]
\[y^2 = x^3 + 8x\]
\[y^2 = x^3 + 9x\]
\[y^2 = x^3 + 10x + 3\]
\[y^2 = x^3 + 10x + 14\]
\[y^2 = x^3 + 11x + 6\]
\[y^2 = x^3 + 11x + 11\]
\[y^2 = x^3 + 12x + 2\]
\[y^2 = x^3 + 12x + 15\]
\[y^2 = x^3 + 13x + 5\]
\[y^2 = x^3 + 13x + 12\]
\[y^2 = x^3 + 14x + 1\]
\[y^2 = x^3 + 14x + 16\]
\[y^2 = x^3 + 15x\]
\[y^2 = x^3 + 16x + 7\]
\[y^2 = x^3 + 16x + 10\]
Pa kao riješenje mogu uzeti bilo koju od tih krivulja, na primjer $E \, : \; y^2=x^3+x+6$.
\newpage
\subsection{3. zadatak}
U trećem zadatku sam koristio algoritam naveden u \cite[s. ~60]{ekk}. Implementirao sam ga također pomoću $Sagea$ i $Phytona$ te je njegova implementacija navedena:
\lstinputlisting[language=Python]{EKKDZ4_3.py}
Pa vrijedi da je $|E(\mathbb{F}_{191})|=208$.
 

\end{spacing}
\newpage
\nocite{*}
\bibliographystyle{abbrv}
\bibliography{lit}
\end{document}